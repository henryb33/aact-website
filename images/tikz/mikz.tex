\usepackage{rotating}
\usepackage{tikz}
\usetikzlibrary{patterns}
\usetikzlibrary{math}
\usetikzlibrary{scopes}
\usetikzlibrary{fadings}
\usetikzlibrary{arrows}
\usetikzlibrary{decorations,decorations.text,backgrounds}
\tikzset{>=latex}

% NOT BEAMER

% \pgfdeclarelayer{bg}    % declare background layer
% \pgfsetlayers{bg,main}  % set the order of the layers (main is the standard layer)

\definecolor{beige}{RGB}{245,245,220}
\definecolor{darkred}{rgb}{0.90,0.00,0.00}%red}%!75!black!100}
\definecolor{darkgreen}{rgb}{0.00,0.45,0.00}
% \definecolor{uwgold}{RGB}{232,211,162}
\definecolor{uwgold}{HTML}{b7a57a}
\definecolor{uwpurp}{RGB}{75,47,132}
\definecolor{uwdarkgold}{RGB}{184,165,122}

\colorlet{spcol}{blue!35!white!95!black}

\newcommand{\threeaxes}[8]{
	\tikzmath{
    	\Lxr=#3;\Lxl=#3;\Lyt=#4;\Lyb=#4;\Zt= #5;\Zb= #6;\Xang=#7;\Yang=#8;
    	\Xxr= cos(\Xang)*\Lxr; \Xyr=-sin(\Xang)*\Lxr;
    	\Xxl=-cos(\Xang)*\Lxl; \Xyl= sin(\Xang)*\Lxl;
        \Yxt= sin(\Yang)*\Lyt; \Yyt= cos(\Yang)*\Lyt;
        \Yxb=-sin(\Yang)*\Lyb; \Yyb=-cos(\Yang)*\Lyb;
        \zzz=0;
    }
    \begin{scope}[shift={(#1,#2)},rotate=0]
        \ifx\Zt\zzz\else  \draw[black,->] (0,0) -- +(0, \Zt);    \fi
        \ifx\Zb\zzz\else  \draw[black,->] (0,0) -- +(0,-\Zb);    \fi
        \ifx\Lxr\zzz\else \draw[black,->] (0,0) -- +(\Xxr,\Xyr); \fi
        \ifx\Lxl\zzz\else \draw[black,->] (0,0) -- +(\Xxl,\Xyl); \fi
        \ifx\Lyt\zzz\else \draw[black,->] (0,0) -- +(\Yxt,\Yyt); \fi
        \ifx\Lyb\zzz\else \draw[black,->] (0,0) -- +(\Yxb,\Yyb); \fi
	\end{scope}
}
\newcommand{\threeaxeslabelx}[9]{
	\tikzmath{
    	\Lxr=#3;\Lxl=#3;\Lyt=#4;\Lyb=#4;\Zt= #5;\Zb= #6;\Xang=#7;\Yang=#8;
    	\Xxr= cos(\Xang)*\Lxr; \Xyr=-sin(\Xang)*\Lxr;
    	\Xxl=-cos(\Xang)*\Lxl; \Xyl= sin(\Xang)*\Lxl;
        \Yxt= sin(\Yang)*\Lyt; \Yyt= cos(\Yang)*\Lyt;
        \Yxb=-sin(\Yang)*\Lyb; \Yyb=-cos(\Yang)*\Lyb;
        \zzz=0;
    }
    \begin{scope}[shift={(#1,#2)},rotate=0]
%        \ifx\Zt\zzz\else  \draw (0,\Zt)     node[anchor=south west] {#9}; \fi
        \ifx\Lxr\zzz\else \draw (\Xxr,\Xyr) node[anchor=south west] {#9}; \fi
%        \ifx\Lyt\zzz\else \draw (\Yxt,\Yyt) node[anchor=south west] {#9}; \fi
	\end{scope}
}
\newcommand{\threeaxeslabely}[9]{
	\tikzmath{
    	\Lxr=#3;\Lxl=#3;\Lyt=#4;\Lyb=#4;\Zt= #5;\Zb= #6;\Xang=#7;\Yang=#8;
    	\Xxr= cos(\Xang)*\Lxr; \Xyr=-sin(\Xang)*\Lxr;
    	\Xxl=-cos(\Xang)*\Lxl; \Xyl= sin(\Xang)*\Lxl;
        \Yxt= sin(\Yang)*\Lyt; \Yyt= cos(\Yang)*\Lyt;
        \Yxb=-sin(\Yang)*\Lyb; \Yyb=-cos(\Yang)*\Lyb;
        \zzz=0;
    }
    \begin{scope}[shift={(#1,#2)},rotate=0]
%        \ifx\Zt\zzz\else  \draw (0,\Zt)     node[anchor=south west] {#9}; \fi
%        \ifx\Lxr\zzz\else \draw (\Xxr,\Xyr) node[anchor=south west] {#9}; \fi
        \ifx\Lyt\zzz\else \draw (\Yxt,\Yyt) node[anchor=south west] {#9}; \fi
	\end{scope}
}
\newcommand{\threeaxeslabelz}[9]{
	\tikzmath{
    	\Lxr=#3;\Lxl=#3;\Lyt=#4;\Lyb=#4;\Zt= #5;\Zb= #6;\Xang=#7;\Yang=#8;
    	\Xxr= cos(\Xang)*\Lxr; \Xyr=-sin(\Xang)*\Lxr;
    	\Xxl=-cos(\Xang)*\Lxl; \Xyl= sin(\Xang)*\Lxl;
        \Yxt= sin(\Yang)*\Lyt; \Yyt= cos(\Yang)*\Lyt;
        \Yxb=-sin(\Yang)*\Lyb; \Yyb=-cos(\Yang)*\Lyb;
        \zzz=0;
    }
    \begin{scope}[shift={(#1,#2)},rotate=0]
        \ifx\Zt\zzz\else  \draw (0,\Zt)     node[anchor=west] {$\;$#9}; \fi
%        \ifx\Lxr\zzz\else \draw (\Xxr,\Xyr) node[anchor=south west] {#9}; \fi
%        \ifx\Lyt\zzz\else \draw (\Yxt,\Yyt) node[anchor=south west] {#9}; \fi
	\end{scope}
}

\newcommand{\isoaxes}[9]{
    \tikzmath{
        \rot=#3;
        \len=#4;
        \ddd=#5;
        \pX=  0; \Xx=cos(\pX)*\len; \Xy=sin(\pX))*\len;
        \pY=120; \Yx=cos(\pY)*\len; \Yy=sin(\pY))*\len;
        \pZ=240; \Zx=cos(\pZ)*\len; \Zy=sin(\pZ))*\len;
        \Xxx=cos(\ddd)*\Xx-sin(\ddd)*\Xy; \Xxy=sin(\ddd)*\Xx+cos(\ddd)*\Xy;
        \Yxx=cos(\ddd)*\Yx-sin(\ddd)*\Yy; \Yxy=sin(\ddd)*\Yx+cos(\ddd)*\Yy;
        \Zxx=cos(\ddd)*\Zx-sin(\ddd)*\Zy; \Zxy=sin(\ddd)*\Zx+cos(\ddd)*\Zy;
    }
    \begin{scope}[shift={(#1,#2)},rotate=\rot]
		\filldraw[black] (0,0) circle (2pt);
        \draw[black,thick,->] (0,0) -- +(\Xx,\Xy);
        \draw[black,thick,->] (0,0) -- +(\Yx,\Yy);
        \draw[black,thick,->] (0,0) -- +(\Zx,\Zy);
        \draw (0.1,0.6)   node[rotate=0,anchor=center] {#6};
        \draw (\Xxx,\Xxy) node[rotate=0,anchor=center] {#7};
        \draw (\Yxx,\Yxy) node[rotate=0,anchor=center] {#8};
        \draw (\Zxx,\Zxy) node[rotate=0,anchor=center] {#9};
    \end{scope}
}

\newcommand{\isoaxesNL}[6]{
    \tikzmath{
        \rot=#3;
        \len=#4;
        \ddd=#5;
        \pX=  0; \Xx=cos(\pX)*\len; \Xy=sin(\pX))*\len;
        \pY=120; \Yx=cos(\pY)*\len; \Yy=sin(\pY))*\len;
        \pZ=240; \Zx=cos(\pZ)*\len; \Zy=sin(\pZ))*\len;
        \Xxx=cos(\ddd)*\Xx-sin(\ddd)*\Xy; \Xxy=sin(\ddd)*\Xx+cos(\ddd)*\Xy;
        \Yxx=cos(\ddd)*\Yx-sin(\ddd)*\Yy; \Yxy=sin(\ddd)*\Yx+cos(\ddd)*\Yy;
        \Zxx=cos(\ddd)*\Zx-sin(\ddd)*\Zy; \Zxy=sin(\ddd)*\Zx+cos(\ddd)*\Zy;
    }
    \begin{scope}[shift={(#1,#2)},rotate=\rot]
		\filldraw[black] (0,0) circle (2pt);
        \draw[black,thick,->] (0,0) -- +(\Xx,\Xy);
        \draw[black,thick,->] (0,0) -- +(\Yx,\Yy);
        \draw[black,thick,->] (0,0) -- +(\Zx,\Zy);
        \draw (0.1,0.6)   node[rotate=0,anchor=center] {#6};
    \end{scope}
}

\newcommand{\isoaxesNLC}[7]{
    \tikzmath{
        \rot=#3;
        \len=#4;
        \ddd=#5;
        \pX=  0; \Xx=cos(\pX)*\len; \Xy=sin(\pX))*\len;
        \pY=120; \Yx=cos(\pY)*\len; \Yy=sin(\pY))*\len;
        \pZ=240; \Zx=cos(\pZ)*\len; \Zy=sin(\pZ))*\len;
        \Xxx=cos(\ddd)*\Xx-sin(\ddd)*\Xy; \Xxy=sin(\ddd)*\Xx+cos(\ddd)*\Xy;
        \Yxx=cos(\ddd)*\Yx-sin(\ddd)*\Yy; \Yxy=sin(\ddd)*\Yx+cos(\ddd)*\Yy;
        \Zxx=cos(\ddd)*\Zx-sin(\ddd)*\Zy; \Zxy=sin(\ddd)*\Zx+cos(\ddd)*\Zy;
    }
    \begin{scope}[shift={(#1,#2)},rotate=\rot]
		\filldraw[#7] (0,0) circle (2pt);
        \draw[#7,thick,->] (0,0) -- +(\Xx,\Xy);
        \draw[#7,thick,->] (0,0) -- +(\Yx,\Yy);
        \draw[#7,thick,->] (0,0) -- +(\Zx,\Zy);
        \draw (0.1,0.6)   node[#7,rotate=0,anchor=center] {#6};
    \end{scope}
}

\newcommand{\centerofmass}[1]{%
    \tikz[radius=0.4em,scale=#1] {%
        \fill (0,0) -- ++(0.4em,0) arc [start angle=0,end angle=90] -- ++(0,-0.8em) arc [start angle=270, end angle=180];%
        \draw (0,0) circle;%
    }%
}

\newcommand{\pane}[7]{
	\tikzmath{
    	\rot=#3;
    	\width=#4;
        \height=#5;
        \corner=#6;
    	\px1= 0.5*\width-\corner; \py1=-0.5*\height;
        \px2= 0.5*\width;         \py2=-0.5*\height+\corner;
        \px3= 0.5*\width;         \py3= 0.5*\height-\corner;
        \px4= 0.5*\width-\corner; \py4= 0.5*\height;
        \px5=-0.5*\width+\corner; \py5= 0.5*\height;
        \px6=-0.5*\width;         \py6= 0.5*\height-\corner;
        \px7=-0.5*\width;         \py7=-0.5*\height+\corner;
        \px8=-0.5*\width+\corner; \py8=-0.5*\height;
    }
	\begin{scope}[shift={(#1,#2)},rotate=\rot]
    	\filldraw[{#7}]
        (\px1,\py1) to[out=    0,in=  -90] (\px2,\py2) --
        (\px3,\py3) to[out=   90,in=    0] (\px4,\py4) --
        (\px5,\py5) to[out= -180,in=   90] (\px6,\py6) --
        (\px7,\py7) to[out=  -90,in= -180] (\px8,\py8) -- cycle;
    \end{scope}
}
\newcommand{\cpane}[7]{
	\tikzmath{
		\pLTx=#1;
		\pRBx=#2;
		\pLTy=#3;
		\pRBy=#4;
	}
	\pane{0.5*\pLTx+0.5*\pRBx}{0.5*\pLTy+0.5*\pRBy}{#5}{\pRBx-\pLTx}{\pLTy-\pRBy}{#6}{#7}
}

\newcommand{\coneback}[7]{
	\tikzmath{\rot=#3;
              \length=#4; \radius=\length*tan(0.5*#5); \depth=#6;
              \sx =  cos(\rot)*#1 + sin(\rot)*#2;
              \sy = -sin(\rot)*#1 + cos(\rot)*#2;
    }
    \begin{scope}[shift={(\sx,\sy)},transform canvas={rotate=\rot}]
	    \draw[{#7}] (\radius,-\length) arc(360:180: {\radius} and {-\depth});
    \end{scope}
}

\newcommand{\cone}[7]{
	\tikzmath{\rot=#3;
              \length=#4; \radius=\length*tan(0.5*#5); \depth=#6;
              \sx =  cos(\rot)*#1 + sin(\rot)*#2;
              \sy = -sin(\rot)*#1 + cos(\rot)*#2;
	}
    \begin{scope}[shift={(\sx,\sy)},transform canvas={rotate=\rot}]
    	\fill[{#7}] (0,0) -- (\radius,-\length) arc(360:180: {\radius} and {\depth}) -- cycle;
    	\draw[color=black!100] (0,0) -- (\radius,-\length) arc(360:180: {\radius} and {\depth}) -- cycle;
    \end{scope}
}

\newcommand{\rocket}[6]{
	\tikzmath{\rot=#3;
    		  \length=#4;
              \throttle=(\length/0.8)*0.6*#5;
              \gimbalangle=#6;
    		  \LL = \length; \RR = \LL/8;
              \HH = \LL/5;   \WW = \LL/16;
              \rr = \LL/16;  \ww = \LL/16;
              \HG = \LL/8;   \RG = \LL/11; \WG = \LL/16;
              \sx =  cos(\rot)*#1 + sin(\rot)*#2;
              \sy = -sin(\rot)*#1 + cos(\rot)*#2;
    }
    \begin{scope}[shift={(\sx,\sy)},transform canvas={rotate=\rot}]
		\begin{scope}[shift={(0,-0.5*\LL)},rotate=\gimbalangle]
          	% Flame %
          	\fill[fill=orange!100,shading=axis,shading angle=90,left color=orange!100,right color=orange!25]
            	(-\RG,-\HG) -- (0,-\HG-\throttle) -- (\RG,-\HG) -- cycle;
        	% Gimbal %
        	\filldraw[color=black!100,fill=black!10,shading=axis,shading angle=90,left color=black!30,right color=black!0]
        	(0,0) -- (\RG,-\HG) arc(360:180: {\RG} and {\WG}) -- cycle;
		\end{scope}
        
        % Rocket Body %
    	\filldraw[color=black!100,fill=black!10,shading=ball,shading angle=90,left color=black!30,right color=black!0]
        	(-\RR,-0.5*\LL) arc(180:360: {\RR} and {\WW}) -- (\RR,0.5*\LL) -- (\RR,0.5*\LL) arc(360:180: {\RR} and {\WW}) -- (-\RR,-0.5*\LL) -- cycle;
        
        % Rocket Nose %
    	\filldraw[color=black!100,fill=black!10,shading=axis,shading angle=90,left color=black!30,right color=black!0]
        	(\RR,0.5*\LL) arc(360:180: {\RR} and {\WW}) -- (-\rr,0.5*\LL+\HH) arc(-180:0: {\rr} and {-\ww}) -- cycle;
    \end{scope}
}

\newcommand{\munderbrace}[7]{
	\tikzmath{
%		\cx = #1;
%		\cy = #2;
%		\rot = #3;
%		\width = #4;
%		\heightb = #5;
%		\heightt = #6;
	}
	\begin{scope}[shift={(#1,#2)},rotate=#3]
		\draw [{#7}] (0,0) -- (0,#5);
		\draw [{#7}] (0.5*#4,#5+#6) -- (0.5*#4,#5) -- (-0.5*#4,#5) -- (-0.5*#4,#5+#6);
	\end{scope}
}

\newcommand{\quadrotor}[6]{
	\tikzmath{
		\Larm = 1; \Rr = 0.9;
		%\Rrx = \Rr; \Rry = \Rr; 
		\rx1= \Larm; \ry1= \Larm;
		\rx2= \Larm; \ry2=-\Larm;
		\rx3=-\Larm; \ry3=-\Larm;
		\rx4=-\Larm; \ry4= \Larm;
		\drx1=\rx1+\Rr*cos(45); \dry1=\ry1+\Rr*sin(45);
		\val3 = #3;
		\val2 = #4;
		\val1 = #5;
	}
	\begin{scope}[shift={(#1,#2)}]
		\begin{scope}[transform canvas={scale=#6}]
			\shade [ball color=white] (0,0) circle (5pt);
			\draw [black,line width=0.5pt] (0,0) circle (5pt);
			
			\begin{scope}[rotate=\val1]
				\begin{scope}[xscale=\val2]
					\begin{scope}[rotate=-\val3]
%						\begin{scope}[yscale=\val3]
							\begin{scope}[shift={(\rx1,\ry1)}]
								\fill[shading=radial,outer color=orange!25,inner color=orange!100] (0,0) circle (\Rr);
								\draw[-,>=stealth,line cap=round,line width=2pt] ( 90:\Rr) arc ( 90:-270:\Rr);
							\end{scope}
							\begin{scope}[shift={(\rx2,\ry2)}]
								\fill[shading=radial,outer color=black!5,inner color=black!25] (0,0) circle (\Rr);
								\draw[-,>=stealth,line cap=round,line width=2pt] (-90:\Rr) arc (-90: 270:\Rr);
							\end{scope}
							\begin{scope}[shift={(\rx3,\ry3)}]
								\fill[shading=radial,outer color=black!5,inner color=black!25] (0,0) circle (\Rr);
								\draw[-,>=stealth,line cap=round,line width=2pt] (-90:\Rr) arc (-90: 270:\Rr);
							\end{scope}
							\begin{scope}[shift={(\rx4,\ry4)}]
								\fill[shading=radial,outer color=orange!25,inner color=orange!100] (0,0) circle (\Rr);
								\draw[-,>=stealth,line cap=round,line width=2pt] ( 90:\Rr) arc ( 90:-270:\Rr);
							\end{scope}
						\end{scope}
%					\end{scope}
				\end{scope}
			\end{scope}
			
%			\begin{scope}[rotate=-90]
%				\begin{scope}[xscale=\val1]
%					\begin{scope}[rotate=-\val2]
%						\begin{scope}[yscale=\val3]
%							\draw [black,->,line width=0.75pt] (0,0) -- +(0,3);% node[anchor=south west] {$x$};
%							\draw [black,->,line width=0.75pt] (0,0) -- +(3,0);% node[anchor=south west] {$y$};
%						\end{scope}
%					\end{scope}
%				\end{scope}
			
		\end{scope}
	\end{scope}
}

\newcommand{\backwall}[8] {
	\tikzmath {
			\h1=#1; \w1=#2; \a1=#3;
			\h2=#4; \w2=#5; \a2=#6;
			\ax1=0.0;                   \ay1=0.5*\h2;
			\ax2=\ax1;                  \ay2=0.5*\h1;
			\ax3=\ax2-0.5*cos(\a1)*\w1; \ay3=\ay2+0.5*sin(\a1)*\w1;
			\ax4=\ax3;                  \ay4=\ay3-\h1;
			\ax5=\ax4+0.5*cos(\a1)*\w1; \ay5=\ay4-0.5*sin(\a1)*\w1;
			\ax6=\ax5;                  \ay6=-\ay1;
			\ax7=\ax6-cos(\a2)*\w2;     \ay7=\ay6+sin(\a2)*\w2;
			\ax8=\ax7;                  \ay8=\ay7+\h2;
	}
	
	\fill[#7] (\ax1,\ay1)
				 -- (\ax2,\ay2)
				 -- (\ax3,\ay3)
				 -- (\ax4,\ay4)
				 -- (\ax5,\ay5)
				 -- (\ax6,\ay6)
				 -- (\ax7,\ay7)
				 -- (\ax8,\ay8)
				 -- cycle;
	\draw[#8] (\ax2,\ay2)
 				 -- (\ax3,\ay3)
				 -- (\ax4,\ay4)
				 -- (\ax5,\ay5);
	\draw[#8] (\ax6,\ay6)
				 -- (\ax7,\ay7)
				 -- (\ax8,\ay8)
				 -- (\ax1,\ay1);
}

\newcommand{\forewall}[8] {
	\tikzmath {
			\h1=#1; \w1=#2; \a1=#3;
			\h2=#4; \w2=#5; \a2=#6;
			\ax1=0.0;                   \ay1=0.5*\h2;
			\ax2=\ax1;                  \ay2=0.5*\h1;
			\ax3=\ax2-0.5*cos(\a1)*\w1; \ay3=\ay2+0.5*sin(\a1)*\w1;
			\ax4=\ax3;                  \ay4=\ay3-\h1;
			\ax5=\ax4+0.5*cos(\a1)*\w1; \ay5=\ay4-0.5*sin(\a1)*\w1;
			\ax6=\ax5;                  \ay6=-\ay1;
			\ax7=\ax6-cos(\a2)*\w2;     \ay7=\ay6+sin(\a2)*\w2;
			\ax8=\ax7;                  \ay8=\ay7+\h2;
	}
	
	\fill[#7] (-\ax1,-\ay1)
				 -- (-\ax2,-\ay2)
				 -- (-\ax3,-\ay3)
				 -- (-\ax4,-\ay4)
				 -- (-\ax5,-\ay5)
		     -- (-\ax6,-\ay6)
				 -- (-\ax7,-\ay7)
				 -- (-\ax8,-\ay8)
				 -- cycle;
	\draw[#8] (-\ax2,-\ay2)
				 -- (-\ax3,-\ay3)
 				 -- (-\ax4,-\ay4)
 				 -- (-\ax5,-\ay5);
	\draw[#8] (-\ax6,-\ay6)
				 -- (-\ax7,-\ay7)
				 -- (-\ax8,-\ay8)
				 -- (-\ax1,-\ay1);
}

\newcommand{\satelliteUUU}[4]{
    % \satelliteUUU{cx}{cy}{rotate}{scale}
    \tikzmath{ \lx=1; \ly=1; \lz=3;
        \lzx=\lz*cos(30); \lzy=\lz*sin(30);
        \lxx=\lx*cos(30); \lxy=\lx*sin(30);
        \ca=cos(30); \sa=sin(30);
        \aa=#3;
        \sr=0.5;
        \xr=\sr*cos(45); \yr=\sr*sin(45);
        \sx=\lx; \sy=\ly;
        \Lcone=1.5; \Acone=2*45; \Dcone=0.25;
        \xp=1/sqrt(2.75); \yp=1.5/sqrt(2.75);
        \cx=#1; \cy=#2;
        \dx=0.1;\dy=0.05; \dsep=0.59; }

% Main satellite
\begin{scope}[shift={(\cx,\cy)},rotate=\aa,scale=#4]
\shadedraw[thick,fill=black!5!white,rounded corners=1pt,left color=black!10!white,right color=black!25!white] (0,0) -- (0,\ly) -- (\lzx,\lzy+\ly) -- (\lzx,\lzy) -- cycle;
\shadedraw[thick,fill=black!5!white,rounded corners=1pt,left color=yellow!10!white,right color=black!10!white] (0,0) -- (0,\ly) -- (-\lxx,\lxy+\ly) -- (-\lxx,\lxy) -- cycle;
\shadedraw[thick,fill=white,rounded corners=1pt,left color=yellow!10!white,right color=black!25!white] (0,\ly) -- (\lzx,\lzy+\ly) -- +(-\lxx,\lxy) -- (-\lxx,\lxy+\ly) -- cycle;

% TOP SOLAR PANELS
\draw[spcol,fill=blue!15!white] (-\lxx+\dx,\lxy+\ly) -- ++(0.35,0.2) -- ++(0.2,0.02) -- ++(0.6*\ca,-0.6*\sa) -- ++(0.0,-0.18) -- ++(-0.3,-0.18) -- cycle;
\draw[spcol,fill=blue!15!white] (-\lxx+\dx+\dsep*\ca,\lxy+\ly+\dsep*\sa) -- ++(0.35,0.2) -- ++(0.2,0.02) -- ++(0.6*\ca,-0.6*\sa) -- ++(0.0,-0.18) -- ++(-0.3,-0.18) -- cycle;
\draw[spcol,fill=blue!15!white] (-\lxx+\dx+2*\dsep*\ca,\lxy+\ly+2*\dsep*\sa) -- ++(0.35,0.2) -- ++(0.2,0.02) -- ++(0.6*\ca,-0.6*\sa) -- ++(0.0,-0.18) -- ++(-0.3,-0.18) -- cycle;
% \draw[spcol,fill=blue!15!white] (-\lxx+\dx+3*\dsep*\ca,\lxy+\ly+3*\dsep*\sa) -- ++(0.35,0.2) -- ++(0.2,0.02) -- ++(0.6*\ca,-0.6*\sa) -- ++(0.0,-0.18) -- ++(-0.3,-0.18) -- cycle;
\draw[spcol,fill=blue!15!white] (-\lxx+\dx+4*\dsep*\ca,\lxy+\ly+4*\dsep*\sa) -- ++(0.35,0.2) -- ++(0.2,0.02) -- ++(0.6*\ca,-0.6*\sa) -- ++(0.0,-0.18) -- ++(-0.3,-0.18) -- cycle;

% SIDE SOLAR PANELS
\draw[spcol,fill=blue!15!white] (0.5*\dx,1.5*\dy) -- ++(0,\ly-2*\dy) -- ++(0.35*\ca,0.35*\sa) -- ++(0.15,-0.1) -- ++(0,-0.6) -- ++(-0.1,-0.17) -- cycle;
\draw[spcol,fill=blue!15!white] (0.5*\dx+\dsep*\ca,1.5*\dy+\dsep*\sa) -- ++(0,\ly-2*\dy) -- ++(0.35*\ca,0.35*\sa) -- ++(0.15,-0.1) -- ++(0,-0.6) -- ++(-0.1,-0.17) -- cycle;
\draw[spcol,fill=blue!15!white] (0.5*\dx+2*\dsep*\ca,1.5*\dy+2*\dsep*\sa) -- ++(0,\ly-2*\dy) -- ++(0.35*\ca,0.35*\sa) -- ++(0.15,-0.1) -- ++(0,-0.6) -- ++(-0.1,-0.17) -- cycle;
\draw[spcol,fill=blue!15!white] (0.5*\dx+3*\dsep*\ca,1.5*\dy+3*\dsep*\sa) -- ++(0,\ly-2*\dy) -- ++(0.35*\ca,0.35*\sa) -- ++(0.15,-0.1) -- ++(0,-0.6) -- ++(-0.1,-0.17) -- cycle;
\draw[spcol,fill=blue!15!white] (0.5*\dx+4*\dsep*\ca,1.5*\dy+4*\dsep*\sa) -- ++(0,\ly-2*\dy) -- ++(0.35*\ca,0.35*\sa) -- ++(0.15,-0.1) -- ++(0,-0.6) -- ++(-0.1,-0.17) -- cycle;

% END PLATE
\draw[blue!35!white,fill=black!10!white,rounded corners=1pt] (-0.5*\dx,2*\dy) -- (-0.5*\dx,\ly-\dy) -- (-\ca+0.5*\dx,\ly+\sa-2*\dy) -- (-\ca+0.5*\dx,\sa+\dy) -- cycle;
\draw[black,fill=black!50!white] (-0.5*\lx,0.75*\ly) circle (2pt);
\end{scope}
}
